\section{Client}
\label{sec:client}

The client exists to accept messages sent by the server, and present them in an
order and a format that the user interface can present to the user, as well as
accept the information entered by the user into the user interface and pass it
to the server.

\subsection{System Design}
\label{sub:client_system_design}

\begin{enumerate}

	\item \textbf{Login}

	When a client is started, a \texttt{QuizClient} object is created and
	starts the main loop.  The initial stages set up the connection with the
	server and waits for the user to login. When the user enters their
	information, a \texttt{LoginRequest} object is created and pass to the
	server containing the username and password, to be checked against the
	contents of the database. The client then waits for a reponse from the
	server to indicate if the login was successfull or not. This comes in the
	form of a \texttt{LoginReply} object. If this says that the login was
	unsuccessful, the user is asked to re-input their details, otherwise, the
	display is changed and the options screen is shown.

	\item \textbf{Start}

	From here, the user can select the ``Start Quiz'' option to start the
	quiz. This causes the display to change to display the waiting screen and
	the client waits for information from the server.

	From this point on, the client waits for any object to be sent from the
	server and acts according to what that object was. The possible objects
	that the client now expects to be able to distinguish between are:
	\begin{itemize}
		\item \texttt{Quiz}
		\item \texttt{StartQuiz}
		\item \texttt{DisplayQuestion}
		\item \texttt{Score}
	\end{itemize}

	\begin{description}
		\item[\texttt{Quiz}]\hfill \\ This object contains the information
		about the quiz itself, the number of questions, their contents and the
		duration that each question should be displayed for. It should only
		ever be recieved once by the client, at the start of the session,
		reducing the transfer of information over the connection.

		\item[\texttt{StartQuiz}]\hfill \\ Once an Admin has logged in, they
		have control over the start of the quiz. When they decide to start the
		quiz, this object is sent to each of the listening clients and so the
		client will procede to show the first question from the \texttt{Quiz}
		object.

		\item[\texttt{DisplayQuestion}]\hfill \\ The first question is
		displayed as soon as the \texttt{StartQuiz} object is recieved. After
		this point in the quiz, the questions are changed when this object is
		received. The value contained verifies which question is to be
		displayed.

		\item[\texttt{Score}]\hfill \\ After each question has been answered,
		the client can display a leader board showing the score of all the
		clients that have so far answered the current question along with the
		current client's position in this list. This object tells the client
		the relevant information for displaying the scores of the other
		clients.
	\end{description}

\end{enumerate}
