\section{Protocol}
\label{sec:protocol}

The protocol for communicating between the different parts of the system is
based around objects. There exists an object class that can be created for any
of the several message types that could be needed to transfer information from
the server to the client, or from the client to the server. Each of these
objects implements the \texttt{Serializable} interface, allowing them to be
converted to bytes and transfered across the socket connection.

\begin{description}
	\item[AnswerResponse object] \hfill \\ To respond to a question, the
		Student selects the desired option. This information is passed to the
		server using an AnswerResponse object which simply holds the response
		and the timestamp to indicate when they made the selection.

	\item[DisplayQuestion object]\hfill \\ In order to signify that the
		allotted time for the current question has ended, and the next question
		should be displayed, the server sends a DisplayQuestion object to all
		of the clients and they should move on to the next question in the Quiz
		object and change the GUI accordingly.

	\item[LoginReply object]\hfill \\ Once a loginRequest has been received by
		the server, a LoginReply will be sent back. This gives the client the
		information about the requested login, most importantly if the login
		was successful, as well as the type of user that made the login,
		Student or Admin. This information is used to display the correct user
		interface.

	\item[LoginRequest object]\hfill \\ This is the first object that could be
		created.  It is sent, by the client, to the server and contains the
		username of the Student that is attempting to login and the
		\texttt{java.lang.String} hash code of the inserted password. Though
		the security concerns of such a trivial system are non-existent, the
		password is never stored in plaintext.

	\item[Question object]\hfill \\ There exist several question objects in
		each Quiz object. They store the information required to present a
		Student with a question and the possible answers. Again, there is very
		little functionality as the question only serves as a wrapper for the
		question text and the possible answers that the user could respond
		with.

	\item[Quiz object]\hfill \\ This is the most important object. It has very
		little functionality, simply acting as a wrapper to hold and easily
		transfer several Question objects.

	\item[QuizRequest object]\hfill \\

	\item[Score object]\hfill \\ When the quiz has been completed, each of the
		clients will display the position of it's user relative to the other
		Students. This object contains the score of all of the users so that
		each of the clients can work out where they are in the ranking.

	\item[StartQuiz object]\hfill \\ Once the user has successfully logged into
		the system, the next major event is the start of the quiz. This is
		signaled by an Admin user who is connected to the server, this
		information must then be relayed to each of the connected clients. A
		StartQuiz object is sent to each of the clients, who, on receiving it,
		will display the first question to the user.
\end{description}
