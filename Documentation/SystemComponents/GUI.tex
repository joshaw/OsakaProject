\section{Graphical User Interface}
\label{sec:graphical user interface}

The student and admin users will interact with a graphical user interface to take and 
run the quiz. The model deals with the data so the GUI can just be a visual interaction 
and representation of the client model. Instead of having many pop ups for each of the 
GUI panels, we decided to have one frame per client in which its content panel changes 
depending on the current part of the quiz.

\begin{description}
	\item[Model/View Seperation] \hfill \\ Originally we decided that the GUI would 
		follow a model/view separation format where the model will be separate 
		from the client. This, however, caused some problems so we decided to 
		not have a separate model and use the client as the model instead. One 
		of the problems we encountered was that the GUI would interact with the 
		model and create an object that needs to be sent to the server. We found 
		it tricky to find a way to send objects from the model to the client and 
		realized that there was a lot of redundancy as objects had to go through 
		the model and the client when really there only needed to be one.
	
	\item[GUI Panels] \hfill \\ Each GUI panel used in the quiz has its own class. 
		When a client is created, each of the GUI panels is instantiated and 
		added to an array. To accomplish this we used inheritance. Each of the 
		GUI panels extends the abstract class MasterFrame, and in turn, 
		MasterFrame extends JPanel. Each of the GUI panels is added to the 
		guiElements MasterFrame[]. MasterFrame is an abstract class with one 
		abstract method resetDisplay(). This method is used to reset the 
		contents of the current panel to an updated version of the model. 
		Whenever the content panel of the frame is changed, the resetDisplay() 
		method is called to ensure correct information is displayed.

		The overall GUI will display different panels depending on which part of 
		the system is currently being used. The GUI panels are: 

	\begin{description}

		\item[\texttt{LoginFrame}]\hfill \\ The first panel users will log-in to 
			(both admin and student). LoginFrame contains a username field, 
			a password field and login button. When the login button is 
			activated, it send the contents of the two fields to the client 
			model to create a LoginRequest object using the requestLogin() 
			method.

		\item[\texttt{StudentHomeFrame}]\hfill \\ The panel a student will see 
			once they have logged in / received successful login reply 
			object. Here they can push the Start quiz button which adds them 
			to a pool of students ready to start the quiz. This leads them 
			to the WaitingFrame panel using the requestWaitingScreen() 
			method.

		\item[\texttt{AdminHomeFrame}]\hfill \\ The panel the admin will see 
			once they have logged in / received successful login reply 
			object. Here they can see which students have connected and 
			joined the waiting pool to start the quiz. AdminHomeFrame has a 
			JComboBox which lists the possible quizzes to run. This uses the 
			setCurrentQuizID() method. The admin can then push the start 
			quiz button which sends the selected quiz to all the waiting 
			students using the adminStart() method.

		\item[\texttt{WaitingFrame}]\hfill \\ The panel a student will see when 
			they have pushed start and are waiting for the admin to start 
			the quiz. Contains a message to the student letting them know 
			they are waiting for the admin to start the quiz as well as a 
			JProgressBar in intermediate state.
		
		\item[\texttt{QuestionFrame}]\hfill \\ The panel which displays the 	
			question to the Students. This panel is Is only displayed for a 
			limited time as each question has a short time limit. The 
			question is displayed in a JTextPane, and each of the four 
			possible multiple choice answers is displayed as a JButton 
			beneath. To the side there is a countdown timer which visually 
			displays the time to answer the question. As soon as the 
			question is answered or the time has run out, the panel is 
			changed to the StudentResultsFrame panel. When a button is 
			pushed, the setResponseNumber(i) method is called. If a student 
			doesn't answer, then the response number is set to -1.

		\item[\texttt{StudentResultsFrame}]\hfill \\ Once a student has 
			completed a question they are displayed with this panel. 
			StudentResultsFrame displays the last quesiton with highlighted 
			text for the correct answer and students answer. It also 
			displays the score received for the last question and current 
			leader board for the quiz where each client is ranked by their 
			total score so far. This panel is shown for a set amount of time 
			before leading back to the QuestionFrame for a new question.

		\item[\texttt{AdminResultsFrame}]\hfill \\ The AdminResultsFrame is 
			displayed once the admin has started the quiz. It displays the 
			current question students are completing and a live leader board 
			showing the current scores of the students. The leader board 
			ranks the students by their current total score from the 
			completed questions so far. 

		\item[\texttt{FinalResultsFrame}]\hfill \\ This is similar to 
			StudentResultsFrame but is the final panel the student sees in 
			the quiz. It shows the correct answers to the previous question 
			and the score for the previous question. It also shows the final 
			leader board rankings. Unlike StudentResultsFrame, there is no 
			time limit so the student can see the final results as long as 
			they wish. There is a back button to return the student to the 
			StudentHomeFrame which uses the returnHome() method.
		
	\end{description}

	Other GUI components that we have created and used within the GUI panels are:

	\begin{description}
	
		\item[\texttt{CountDownTimer}]\hfill \\ CountDownTimer is GUI component 
			that counts down seconds. It shows a JProgressBar decrease as 
			the time decreases. Also has a label beneath showing the seconds 
			left to count down. Used in QuestionFrame. 

		\item[\texttt{LeaderBoard}]\hfill \\ A table consisting of each student 
			in the quiz, their position and score. This is updated whenever 
			the resetDislplay() method is called in its parent panel 
			(MasterFrame). Used in StudentResultsFrame, AdminResultsFrame 
			and FinalResultsFrame.
	
	\end{description}
	
\end{description}


