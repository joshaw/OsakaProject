\section{Server}
\label{sec:server}

The server exists to create a link between the database and the quiz program,
as well creating connections with, and processing requests from, clients.

\subsection{System Design}
\label{sub:system_design}

\begin{enumerate}
	\item \textbf{Initialising the server}

	When the server starts, a \texttt{QuizServer} object is created. The object
	creates a connection with the database by taking advantage of the inbuilt 
        Java Database Connectivity (JDBC) functionality, which allows the server 
        object to retrieve, and update, information contained in the database. In 
        addition to this, a \texttt{ServerSocket} object is created, which waits to
        receive connections from \texttt{QuizClient} objects. When a connection from 
        a client is received, a new \texttt{ClientThread} object is created.

	\item \textbf{Database Connectivity}

	The server interacts with the database through static methods contained in
	the \texttt{QuizJDBC} class.  The class allows the server to establish a
	connection with the database through a \texttt{getConnection} method, which
	returns a Connection object.

	A second method, \texttt{isUser}, is called by the server when it receives
	a \texttt{LoginRequest} object from a client thread. The method returns a
	\texttt{LoginReply} object containing the results of the query, which is
	sent to the client.

	The final method, \texttt{getQuiz}, is called when the server receives a
	\texttt{QuizRequest} object from a client, and returns a \texttt{Quiz}
	object.

	\item \textbf{\texttt{ClientThread} Objects}

	When the server establishes a connection with a client, a
	\texttt{ClientThread} object is created, spawning a thread which allows
	interactions to occur between the server and  client. The server then waits
	to receive a \texttt{LoginRequest} from the client, and returns a
	\texttt{LoginReply} object upon receiving it.  Once the client has logged in,
	the server distinguishes between Student and Admin users.

	\item \textbf{Admin Clients}

	If a connection to an Admin has been made, the server waits for a
	\texttt{QuizRequest} object. Once received, the server retrieves the quiz
        in the database through the \texttt{QuizJDBC} class, then distributes this 
        to all connected Student clients.

	\item \textbf{Student Client}

	If a student user has logged in, their instance of \texttt{ClientThread} 
        on the server will wait until the Quiz has been set to ready by an admin. 
        Once a quiz has been started, the server sends the \texttt{Quiz} object to
	all connected student clients, and then waits to receive an \texttt{AnswerReponse}
	from each of the Student client threads. The \texttt{AnswerReposnse} objects 
        contain the clients response to a question, and the time that it took for 
        them to answer it.  The server than calculates the clients score,
	and the results are stored in an ArrayList, which contains the results of
	all connected Student clients.  This ArrayList of scores is sent to
	the connected clients every time it is updated, allowing users to see an
        'updating' results table which contains all student clients scores.

        \item \textbf{Quiz Time}
         
        In order to carry out admin tasks on the server without disrupting the
        main thread which is listening for connecting clients, an inner class
        \texttt{QuizTime} which extends Thread was created. The thread loops through 
        during the quiz session, checking certain conditions, to which it 
        should respond with specific tasks. For example, it checks whether no 
        clients are connected, in which case the quiz stops. It also checks 
        whether the first question of the quiz has finished being asked, in 
        which case further student clients should not be allowed to join the
        quiz.
        
    
        

\end{enumerate}
