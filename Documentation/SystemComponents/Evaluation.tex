\subsection{Evaluation of project work}
\label{sub:evaluation_of_project_work}

\begin{enumerate}

	\item \textbf{Correctness and reliability}
		We set out to create an interactive live educational quiz and accomplished a
		reliable working project. We have had some coding problems throughout the 
		project but worked as team to come up with solutions. One of the major problems 
		we have encountered was updating the student results leader board. Whenever
		a student answers, their score is added	to an array list in the server and this
		should be distributed to every other student so that their results screen is
		accurate. The problem when implementing was that the first student to answer
		would not receive the updated score array list as other students answered.
		This was corrected by having the students constantly listening for new score
		array lists and insured an up to date leader board. This also lead onto other
		problems such as students not being synchronised on the same question. The cause
		of this was as each student requested a new question, the question counter was
		incremented and students skipped questions. This was solved by having a counter
		so that the question would only be incremented once all students had requested the
		new question.
	
	\item \textbf{Performance}
		There were issues with performance with a live quiz that we had to be concerned with.
		After the users have logged	in and the quiz has started, the questions must be 
		synchronised so that each student is answering the same question. This is accomplished 
		through the server sending the question to each client, rather than the client being 
		individually responsible for the current question. The questions are reliably sent to all 
		students at the same time so that quiz is synchronised. When a student answers, they are 
		directed to the results screen where they need to see an updated version of the leader board. 
		As mentioned in \textbf{Correctness and reliability}, a solution for this was found and works well.

	\item \textbf{Usability}
		The system works well and is designed so that users find it easy and intuitive to use.
		We decided on a standardised quiz format where each quiz has 10 questions, and each question
		has 4 multiple choice answers. The design of the GUI is intuitive so that users don't
		have to learn how to use the system. A user should immediately recognise how to use the
		system and not have to rely on previous knowledge to answer quizzes. Originally the GUI for 
		the questions had text boxes displaying the multiple choice answers with buttons next to
		them. We gave the system to fellow students in the lab to test and found that they tended
		to want to push the multiple choice text boxes rather than the buttons. Therefore, we decided
		to scrap the text boxes and put the multiple choice answers directly as the button text.
	
	\item \textbf{Substantiates}
		Much the originally functionality that we set out for we have accomplished.
		The users can successfully log in and are displayed the right screen depending
		on if they are a student or admin. The admin can register students, and
		students can view their profile. The admin can select a quiz and this is
		sent to the student. Students can take the quiz and see a live leader board
		based on the ranking through out the quiz. The admin can also see a leader
		board of the students live results. Extra functionality we would wish to include
		would be for the admin to be able to create quizzes and for these to be stored in
		the database and to be run for students.

\end{enumerate}

\subsection{Evaluation of project process}
\label{sub:evaluation_of_project_process}

\begin{enumerate}

	\item \textbf{Management of team work}

	\item \textbf{Time keeping and scheduling}
		At the start of the project we 
	
	\item \textbf{Subdivision of work}

	\item \textbf{Integration}

	\item \textbf{Summary}

\end{enumerate}
