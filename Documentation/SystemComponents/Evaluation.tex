\subsection{Evaluation of project work}
\label{sub:evaluation_of_project_work}

\begin{enumerate}

	\item \textbf{Correctness and reliability}

		We set out to create an interactive live educational quiz and
		accomplished a reliable working project. We have had some coding
		problems throughout the project but worked as team to come up with
		solutions. One of the major problems we have encountered was updating
		the student results leader board. Whenever a student answers, their
		score is added	to an array list in the server and this should be
		distributed to every other student so that their results screen is
		accurate. The problem when implementing was that the first student to
		answer would not receive the updated score array list as other students
		answered.

		This was corrected by having the students constantly listening for new
		score array lists and insured an up to date leader board. This also
		lead onto other problems such as students not being synchronised on the
		same question. The cause of this was as each student requested a new
		question, the question counter was incremented and students skipped
		questions. This was solved by having a counter so that the question
		would only be incremented once all students had requested the new
		question.

	\item \textbf{Performance}

		There were issues with performance with a live quiz that we had to be
		concerned with. After the users have logged in and the quiz has
		started, the questions must be synchronised so that each student is
		answering the same question. This is accomplished through the server
		sending the question to each client, rather than the client being
		individually responsible for the current question. The questions are
		reliably sent to all students at the same time so that quiz is
		synchronised. When a student answers, they are directed to the results
		screen where they need to see an updated version of the leader board.
		As mentioned in \textbf{Correctness and reliability} above, a solution
		for this was found and works well.

	\item \textbf{Usability}

		The system works well and is designed so that users find it easy and
		intuitive to use. We decided on a standardised quiz format where each
		quiz has 10 questions, and each question has 4 multiple choice answers.
		The design of the GUI is intuitive so that users don't have to learn
		how to use the system. A user should immediately recognise how to use
		the system and not have to rely on previous knowledge to answer
		quizzes. Originally the GUI for the questions had text boxes displaying
		the multiple choice answers with buttons next to them. We gave the
		system to fellow students in the lab to test and found that they tended
		to want to push the multiple choice text boxes rather than the buttons.
		Therefore, we decided to scrap the text boxes and put the multiple
		choice answers directly as the button text.

	\item \textbf{Substantiates}

		Much the originally functionality that we set out for we have
		accomplished.  The users can successfully log in and are displayed the
		right screen depending on if they are a student or admin. The admin can
		register students, and students can view their profile. The admin can
		select a quiz and this is sent to the student. Students can take the
		quiz and see a live leader board based on the ranking through out the
		quiz. The admin can also see a leader board of the students live
		results. Extra functionality we would wish to include would be for the
		admin to be able to create quizzes and for these to be stored in the
		database and to be run for students.

\end{enumerate}

\subsection{Evaluation of project process}
\label{sub:evaluation_of_project_process}

\begin{enumerate}

	\item \textbf{Management of team work}

		Initially the project was split into five sections with each member of
		the group volunteering to be responsible for a section. In addition,
		each member was also assigned a section to assist on. Rather than
		appointing a team leader to distribute the work amongst the group, the
		tasks to be completed were discussed during bi-weekly team meetings.

	\item \textbf{Time keeping and scheduling}

		Group meetings were held twice weekly; one early in the week to discuss
		and distribute the tasks that were to be completed for the following
		week, and a second meeting later in the week to discuss the current
		progress and any problems that were being had. Each task that was to be
		completed was assigned a deadline. Throughout the duration of the
		project, these deadlines were adhered to by all group members.

	\item \textbf{Subdivision of work}

		During the initial stages of the project, each member was assigned
		specific tasks that were to be completed by a certain deadline. By
		adhering to these deadlines, the code for each of the five portions
		were completed within the first two weeks. The remainder of the time
		was then spent on integrating the code and fixing the bugs that
		inhibited the system from meeting the software specification. During
		this time, a list of tasks to be completed was produced and group
		members would systematically work through this list, either
		individually or in smaller subgroups consisting typically of two or
		three members. This required effective communication between each of
		the group members, which was achieved through regular submissions to
		subversion and whole group emails.

	\item \textbf{Integration}

		After the first few weeks of the project we decided to integrate the
		code each team member had written. This took a while to complete and we
		realised that some modifications had to be made.  Once of the
		modifications was changing the client/model/view to model/view
		separation as stated in Section~\ref{sec:graphical_user_interface}. We
		spent a day integrating	different parts of the project where all group
		members were present. This had the advantage of all group members
		knowing about all parts of the system and made the next few weeks of
		working and communication flow much more smoothly.

\end{enumerate}
